This section gives a brief introduction to some foundations of FDA as well as basic
methods in the context of functional responses. The relation and applicability to the
case of densities as responses is discussed.

\subsection{$L^2([a, b])$ and Hilbert Spaces}
\label{sec:l2 and hilbert spaces}
In the context of FDA, it is assumed that the functional data live in
the space of square integrable functions $L^2([a,b])$, with common
support $[a,b]$. This space is a Hilbert space, i.e., a complete inner
product space, equipped with the norm induced by the inner product.
This allows to generalize notions of distance (as the norm induces a
metric), magnitude (given by the norm), and orthogonality (defined by
$\inpr{x}{y} = 0$, with $\inpr{\cdot}{\cdot}$ being the inner product) of elements
in the space from the Euclidean space, in which we usually work, to more
abstract and potentially infinite dimensional spaces, such as function
spaces.

\subsection{Functional Principal Component Analysis}
\label{sec:fpca}
The most popular method for describing structure in our functional data is Functional
Principal Component analysis (FPCA). This is an analogue to Principal Component Analysis
(PCA) in multivariate statistics to the case of infinite dimensions. It builds on the
Karhunen-Loève decomposition (described in the following subsection) to recover functions
that describe the main modes of variation, in descending order.

\subsubsection{Karhunen-Loève Decomposition and Mercer's Theorem}
\label{sec:kh and mercer}
