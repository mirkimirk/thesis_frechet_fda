In this thesis, we delved into the complexities and nuances of Functional Data
Analysis (fda), specifically focusing on the comparative advantages of Fréchet
regression over functional regression methods.

One of the most salient points is the ease of application of the Fréchet methods.
These methods offer a natural way to adapt to the geometry of the underlying function
space, thus providing a more intuitive approach to data analysis in this context.

On the other hand, functional regression comes with its own set of challenges. Notably,
the method is laden with numerous prerequisites and tuning parameters, which make it
both theoretically and practically cumbersome.

The multitude of tuning parameters in functional regression not only complicates its
application but also introduces multiple sources of bias and potential problems. The
method's user-unfriendliness exacerbates these issues, making it less accessible for
broader application.

A possible area for improvement in the simulation study is the possibility to examine
a more complex DGP. \textcite{PanaretosZemel2016} provide some
examples of optimal transport maps that are useful for this process.

Building more inferential tools for the Fréchet regression in Wasserstein context,
\textcite{PetersenLiuDivani2021} developed an approach for computing confidence bands.

In summary, while both Fréchet regression and functional regression have their merits
and demerits, the former's simplicity and adaptability make it a compelling choice for
many applications in fda. The latter, although powerful, suffers from high computational
and theoretical overhead, making it less user-friendly and more prone to bias and errors.

By laying out the mathematical foundations and practical implications of these methods,
this thesis aims to provide a robust framework for future work in this intriguing and
rapidly evolving field.
