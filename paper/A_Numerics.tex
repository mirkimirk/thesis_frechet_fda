Notably during the computation of the Wasserstein-Fréchet mean of a sample of density
functions, we encountered numerical problems. Since the computation of the mean involves
obtaining the quantile densities from the densities, the densities from the small
variance distributions are very close to zero for much of their support, leading to the
qfs to have a very steep slope (i.e. high valued qdfs) near the boundaries. These
astronomical values (even "infs") lead to problems when computing the mean qdf and then
transforming this to a corresponding pdf. The infs are lost values, which lead to skewness.
The following graph shows an extreme and less extreme qdf, as well as our solution with
smaller support:
