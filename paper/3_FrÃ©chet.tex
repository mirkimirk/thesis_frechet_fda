This section will give an introduction to the concept of Fréchet regression
and some background.

\subsection{Fréchet Mean and Variance}
\label{sec:fr mean and variance}



It can be shown that $d_Q(f,g)$ = $d_W(f,g)$, for any two densities $f$ and $g$ in the
space. (Also explain how quantile synchronized mean different and better than naive
cross-sectional mean.)
\begin{lemma}
    \label{lemma:dqeqdw}
    $d_Q = d_W$
\end{lemma}
\begin{proof}
    asdsadasdads
\end{proof}

Then we can see that:
\begin{lemma}
    The Wasserstein-Fréchet mean estimator is given by
    \begin{equation}
        \hat{f}_\oplus(x) = \frac{1}{\hat{q}_\oplus(\hat{F}_\oplus(x))},
    \end{equation}
    with $\hat{q}_\oplus = \frac{1}{n} \sum_{1}^{n} q_i$.
\end{lemma}
\begin{proof}
    By \ref{lemma:dqeqdw}, we can substitute the quantile distance for the
    Wasserstein distance in the computation of the Fréchet mean. We set
    $Q_\oplus(t) = E[Q(t)]$ because $E[Q(t)]$ is the minimizer of
    \begin{equation}
    \label{eq:wf_mean}
        \begin{aligned}
            E[d_w^2(f_i, f_\oplus)]	& =
            E\left[\int_{0}^{1}(F_i^{-1}(t) - F_\oplus^{-1}(t))^2 \,dt\right] \\
                                    & =
            E\left[\int_{0}^{1}(Q_i(t) - Q_\oplus(t))^2 \,dt\right],
        \end{aligned}
    \end{equation}
    see \citet[Chapter~3.1.4]{PanaretosZemel2020}. To compute the corresponding density
    function, use inverse function rule to get
    $f_\oplus(x) = \frac{1}{q_\oplus(F_\oplus(x))}$, which shows that it suffices to
    estimate $q_\oplus$. Remember that
    $q_\oplus = \frac{\mathrm{d}Q_\oplus}{\mathrm{d}t}$. Because of Assumption REFER A1,
    we can pass the differentiation inside the expectation to see
    $q_\oplus = E\left[\frac{\mathrm{d}Q}{\mathrm{d}t}\right]$, which by analogy
    principle suggests to average the sample observed (or previously estimated)
    quantile densities $q_i$ to obtain an estimator $\hat{q}_\oplus$ for $q_\oplus$.
\end{proof}
