This section provides mathematical background for fda, and establishes basic concepts
and notation that will be used throughout this text.

\subsection{Hilbert Space Theory and $\mathcal{L}^2$ Space}
\label{sec:hilbert spaces and l2}
This section will introduce basic mathematical concepts needed for functional data
analysis. In later sections, these will be used for derivations and proofs. It is mostly
based on \textcite[Chapter~2]{HsingEubank2015}.

\subsubsection{Separable Hilbert Spaces}
\label{sec:separable hilbert spaces}
For many methods in fda to work and be theoretically justified, additional structure on
the set of functions under consideration is assumed. This structure is most often the
$\mathcal{L}^2$ space of square integrable functions. This space is a separable Hilbert
space. To define $\mathcal{L}^2$ and the separable Hilbert space, we first need the
following definitions:
\begin{definition}[Inner Product, Inner Product Space, and Orthogonality]
    \label{def:inpr}
    Let \( V \) be a real vector space. A function $\inpr{\cdot}{\cdot}_V$ that maps any
    two vectors \( s, t \in V \) to a real number \( \inpr{s}{t}_V \) is called an
    \textit{inner product} if it satisfies the following properties:
    \begin{enumerate}
        \item \textbf{Symmetry}: \( \inpr{s}{t}_V  = \inpr{t}{s}_V \) for all \( s, t \in V \).
        \item \textbf{Linearity in First Argument}: \( \inpr{as + bt}{w}_V = a \inpr{s}{w}_V + b \inpr{t}{w}_V \) for all \( s, t, w \in V \) and \( a, b \in \mathbb{R} \).
        \item \textbf{Positive Definiteness}: \( \inpr{s}{t}_V \geq 0 \) and \( \inpr{t}{t}_V = 0 \) if and only if \( t = 0 \) for all \( s, t \in V \).
    \end{enumerate}
    A real vector space \( V \) equipped with such an inner product is called an
    \textit{inner product space}. In an inner product space \( V \), two vectors
    \( s, t \in V \) are said to be \textit{orthogonal} if
    $\inpr{s}{t}_V = 0 $.
\end{definition}
To generalize all the notions of interest for fda from Eucliean space to possibly
infinite dimensional spaces, we need the following space:
\begin{definition}[Hilbert Space]
    \label{def:hilbert space}
    Let $\mathcal{H}$ be an inner product space, and denote by $\inpr{\cdot}{\cdot}_\mathcal{H}$
    the inner product of this space. Then $\norm{x}_\mathcal{H} = \sqrt{\inpr{x}{x}_\mathcal{H}}$
    is the norm induced by the inner product, and $d_\mathcal{H}(a, b) = \norm{a - b}_\mathcal{H}$
    is the distance induced by the norm. If $\mathcal{H}$ is complete with respect
    to this metric, i.e., if every Cauchy sequence of elements in $\mathcal{H}$ converges
    in $\mathcal{H}$, then $\mathcal{H}$ is called a \textit{Hilbert space}.
\end{definition}
This concept helps generalizing many useful notions from the finite-dimensional
Euclidean space in multivariate analysis to infinite-dimensional function spaces. The norm
$\norm{\cdot}_\mathcal{H}$ measures the size of an element in $\mathcal{H}$ and makes it
comparable with others. It also induces a metric via
$d_\mathcal{H}(a, b) = \norm{a - b}_\mathcal{H}$ to measure the distance between two
elements in $\mathcal{H}$. The concept of inner product makes it possible to evaluate
the angle between two vectors in $\mathcal{H}$, which is useful for defining projections
and for orthogonal decompositions of this space. Especially the latter will be very
important for dimensionality reduction.

For ..., we need a notion of 
\begin{definition}[Closed Span]
    \label{def:closed span}
    The \textit{closed span} of a subset \( A \) of a Hilbert space \( \mathcal{H} \) is the 
    closure of the linear span of \( A \) with respect to the metric induced by the inner product.
\end{definition}

\begin{definition}[Orthonormal Basis in a Hilbert Space]
    An orthonormal sequence \( \{ e_n \} \) in a Hilbert space \( \mathcal{H} \) is called 
    an \textit{orthonormal basis} if its closed span equals \( \mathcal{H} \).
\end{definition}

\begin{definition}[Separable Hilbert Space]
    A Hilbert space \( \mathcal{H} \) is \textit{separable} if it has an orthonormal basis, 
    which also serves as a countable, dense subset.
\end{definition}

In a separable Hilbert space, the existence of a countable orthonormal basis allows us to 
approximate any element \( f \) in the space arbitrarily closely by finite linear combinations 
of the basis elements. This foundational idea is critical for methods like Functional Principal 
Component Analysis, where we approximate complex functions by projecting them onto a finite 
subset of this basis.



An operator is a functional of elements in this space. It serves as a generalization of
a matrix and will play the same conceptual role in the FDA analogues to mv methods
that matrices do in them.

This is the covariance operator... it is symmetric and positive semi-definite, so
a Hilbert-Schmidt operator (\textcite{WangChiouMüller2016} say because of the integral form,
its a trace class, so compact Hilbert-Schmidt operator). It allows for the spectral
decomposition (in terms of eigenfunctions and eigenvalues). The space of Hilbert-Schmidt
operators is itself a separable Hilbert Space.

\subsubsection{Riesz Representation Theorem}
\label{sec:riesz}

\subsubsection{Mercer's Theorem and Karhunen-Loève Decomposition}
\label{sec:mercer and kh}

\subsubsection{$L^2([a, b])$ Space}
\label{sec:l2 space}
In the context of FDA, it is assumed that the functional data live in
the space of square integrable functions $L^2([a,b])$, with common
support $[a,b]$. This space is a separable Hilbert space, i.e., a complete inner
product space, equipped with the norm induced by the inner product. Separability means
its elements can be arbitrarily well approximated by elements of a dense subset of the
space, and that this subset is not unhandably large.
This allows to generalize notions of distance (as the norm induces a
metric), magnitude (given by the norm), and orthogonality (defined by
$\inpr{x}{y} = 0$, with $\inpr{\cdot}{\cdot}$ being the inner product) of elements
in the space from the Euclidean space, in which we usually work, to more
abstract and potentially infinite dimensional spaces, such as function
spaces.

\begin{definition}[Space of square integrable functions $\mathcal{L}^2$]
    Let $\mathcal{A} \subseteq \mathbb{R}$ be compact. The space \( \mathcal{L}^2(\mathcal{A}) \)
    consists of all functions \( f: \mathcal{A} \to \mathbb{R} \) such that
    \[
    \int_{\mathcal{A}} (f(x))^2 \, dx < \infty
    \]
    Its norm, inner product, and distance are denoted by $\norm{\cdot}_2$,
    $\inpr{\cdot}{\cdot}_2$, and $d_2(\cdot, \cdot)$, respectively.
\end{definition}

\subsection{Fréchet mean and variance}
\label{sec:f_mean}