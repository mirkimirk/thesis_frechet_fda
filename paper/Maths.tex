This section provides mathematical background for fda and the Wasserstein space. Basic
concepts and notation are established that will be used throughout this thesis.

\subsection{Hilbert Space Theory and $\mathcal{L}^2$ Space}
\label{sec:hilbert spaces and l2}
This section will introduce basic mathematical concepts required to explain basic fda
methods. In later sections, these will be used for derivations and proofs. It is mostly
based on \textcite[Chapter~2]{HsingEubank2015}.

\subsubsection{Separable Hilbert Spaces}
\label{sec:separable hilbert spaces}
For many methods in fda to work and be theoretically justified, additional structure on
the set of functions under consideration is assumed. This structure is most often the
$\mathcal{L}^2$ space of square integrable functions. This space is a separable Hilbert
space. To define $\mathcal{L}^2$ and understand its properties as a separable Hilbert
space, we first need the following definitions:
\begin{definition}[Inner Product and Inner Product Space]
    \label{def:inpr}
    Let \( V \) be a real vector space. A function $\inpr{\cdot}{\cdot}_V$ that maps any
    two vectors \( s, t \in V \) to a real number \( \inpr{s}{t}_V \) is called an
    \textit{inner product} if it satisfies the following properties:
    \begin{enumerate}
        \item \textbf{Symmetry}: \( \inpr{s}{t}_V  = \inpr{t}{s}_V \) for all \( s, t \in V \).
        \item \textbf{Linearity in First Argument}: \( \inpr{as + bt}{w}_V = a \inpr{s}{w}_V + b \inpr{t}{w}_V \) for all \( s, t, w \in V \) and \( a, b \in \mathbb{R} \).
        \item \textbf{Positive Definiteness}: \( \inpr{s}{t}_V \geq 0 \) and \( \inpr{t}{t}_V = 0 \) if and only if \( t = 0 \) for all \( s, t \in V \).
    \end{enumerate}
    A real vector space \( V \) equipped with such an inner product is called an
    \textit{inner product space}.
\end{definition}
To generalize all the notions of interest for fda from Eucliean space to possibly
infinite dimensional spaces, we need the following space:
\begin{definition}[Hilbert Space]
    \label{def:hilbert space}
    Let $\mathcal{H}$ be an inner product space, and denote by $\inpr{\cdot}{\cdot}_\mathcal{H}$
    the inner product of this space. Then $\norm{x}_\mathcal{H} = \sqrt{\inpr{x}{x}_\mathcal{H}}$
    is the norm induced by the inner product, and $d_\mathcal{H}(a, b) = \norm{a - b}_\mathcal{H}$
    is the distance induced by the norm. If $\mathcal{H}$ is complete with respect
    to this metric, i.e., if every Cauchy sequence of elements in $\mathcal{H}$ converges
    in $\mathcal{H}$, then $\mathcal{H}$ is called a \textit{Hilbert space}.
\end{definition}
This concept helps generalizing many useful notions from the finite-dimensional
Euclidean space in multivariate analysis to infinite-dimensional function spaces. The norm
$\norm{\cdot}_\mathcal{H}$ measures the size of an element in $\mathcal{H}$ and makes it
comparable with others. It also induces a metric via
$d_\mathcal{H}(a, b) = \norm{a - b}_\mathcal{H}$ to measure the distance between two
elements in $\mathcal{H}$. The concept of inner product makes it possible to evaluate
the angle between two vectors in $\mathcal{H}$, which is useful for defining projections
and for orthogonal decompositions of this space. Especially the latter will be very
important for dimensionality reduction. Orthogonality is defined in the following
- Definition orthogonality, and orthonormal sequence in Inner product space (use Kronecker
delta).

Later we will be approximating elements of a Hilbert space by a combination of a finite
number of its elements. So to find the most efficient subspace (in the sense of giving the
best approximation for a given number of dimensions), we want to exploit some more
structure of the space $\mathcal{L}^2$ that we will be interested in. For this we need an analogue notion of a \textit{linear span}
from the finite dimensional vector spaces we know from multivariate analysis to this
infinite dimensional space. The following definition gives this notion.

Definition of closed span: Let $\mathcal{h}_1,\mathcal{h}_2, \dots $ be a sequence of
elements in 




For ..., we need a notion of 
\begin{definition}[Closed Span, Orthogonality, and Orthonormal Sequence]
    \label{def:closed span}
    The \textit{closed span} of a subset \( A \) of a Hilbert space \( \mathcal{H} \) is the 
    closure of the linear span of \( A \) with respect to the metric induced by the inner product.

    In an inner product space \( V \), two vectors \( s, t \in V \) are said to be
    \textit{orthogonal} if
    $\inpr{s}{t}_V = 0 $.
\end{definition}

\begin{definition}[Orthonormal Basis in a Hilbert Space]
    An orthonormal sequence \( \{ e_n \} \) in a Hilbert space \( \mathcal{H} \) is called 
    an \textit{orthonormal basis} if its closed span equals \( \mathcal{H} \).
\end{definition}

\begin{definition}[Separable Hilbert Space]
    A Hilbert space \( \mathcal{H} \) is \textit{separable} if it has an orthonormal basis, 
    which also serves as a countable, dense subset.
\end{definition}

In a separable Hilbert space, the existence of a countable orthonormal basis allows us to 
approximate any element \( f \) in the space arbitrarily closely by finite linear combinations 
of the basis elements. This foundational idea is critical for methods like Functional Principal 
Component Analysis, where we approximate complex functions by projecting them onto a finite 
subset of this basis.

\subsubsection{$\mathcal{L}^2([a, b])$ Space}
\label{sec:l2 space}
In the context of FDA, it is assumed that the functional data live in
the space of square integrable functions $\mathcal{L}^2([a,b])$, with common
support $[a,b]$. This space is a separable Hilbert space, i.e., a complete inner
product space, equipped with the norm induced by the inner product. Separability means
its elements can be arbitrarily well approximated by elements of a dense subset of the
space, and that this subset is not unhandably large.
This allows to generalize notions of distance (as the norm induces a
metric), magnitude (given by the norm), and orthogonality (defined by
$\inpr{x}{y} = 0$, with $\inpr{\cdot}{\cdot}$ being the inner product) of elements
in the space from the Euclidean space, in which we usually work, to more
abstract and potentially infinite dimensional spaces, such as function
spaces.

\begin{definition}[Space of square integrable functions $\mathcal{L}^2$]
    Let $\mathcal{A} \subseteq \mathbb{R}$ be compact. The space \( \mathcal{L}^2(\mathcal{A}) \)
    consists of all functions \( f: \mathcal{A} \to \mathbb{R} \) such that
    \[
    \int_{\mathcal{A}} (f(x))^2 \, dx < \infty
    \]
    Its norm, inner product, and distance are denoted by $\norm{\cdot}_2$,
    $\inpr{\cdot}{\cdot}_2$, and $d_2(\cdot, \cdot)$, respectively.
\end{definition}

\subsubsection{Linear Operators in Hilbert Spaces}
\label{sec:operators}

An operator is a functional of elements in this space. It serves as a generalization of
a matrix and will play the same conceptual role in the FDA analogues to mv methods
that matrices do in them. This section is mainly based on \textcite[Chapter~]{HorvathKokoszka2012}

This is the covariance operator... it is symmetric and positive semi-definite, so
a Hilbert-Schmidt operator (\textcite{WangChiouMüller2016} say because of the integral form,
its a trace class, so compact Hilbert-Schmidt operator). It allows for the spectral
decomposition (in terms of eigenfunctions and eigenvalues). The space of Hilbert-Schmidt
operators is itself a separable Hilbert Space.

\subsubsection{Important Theorems}
\label{sec:theorems}

\subsubsection{Riesz Representation Theorem}
\label{sec:riesz}
Will help later in interpreting the covariance operator as it is an integral operator
and can be written as an inner product $\inpr{\gamma(x, \cdot)}{\mathcal{h}(\cdot)}$.
Will also be useful finding the solution of Fréchet optimization problem in Wasserstein
space.

\subsubsection{Mercer's Theorem and Karhunen-Loève Decomposition}
\label{sec:mercer and kh}
Mercer makes the calculation of eigenfunctions and values of the covariance operator
useful to represent the covariance function with this. This will be an optimal orthonormal
basis, capturing in descending order the most important directions of variation, so that
when truncated to finite dimensional basis, we have the best possible representation of
elements in the Hilbert space given the truncation parameter.


\subsection{Statistical Foundations in Function Spaces}
\label{sec:stat_foundations}

\subsubsection{Preliminaries and Definitions}
\label{sec:definitions}
A very basic operation we will use throughout this work is switching between different functions
that characterize a distribution. We will deal with the following four important classes
of functions that characterize a distribution:
\begin{definition}[pdfs, cdfs, qfs, and qdfs]
    \label{def:distributionfuncs}
    Let \(\mathbb{R}^+\) be the set of nonnegative reals.
    \begin{enumerate}
        \item The \textit{probability density function (pdf)} is a function
        \( f: \mathbb{R} \to \mathbb{R}^+ \) such that \( \int_{\mathbb{R}} f(x) \, dx = 1 \)
        and \( f(x) \geq 0 \) for all \( x \in \mathbb{R} \).

        \item The \textit{cumulative cistribution function (cdf)} is given by
        \( F(x) \coloneqq \int_{-\infty}^{x} f(t) \, dt \), where \( f \) is the pdf.
        Note that \( F\) is non-decreasing and right-continuous,
        with \( \lim_{{x \to -\infty}} F(x) = 0 \) and \( \lim_{{x \to \infty}} F(x) = 1 \).

        \item The inverse of \(F\) is called the \textit{quantile function (qf)},
        and is denoted by \(Q\). It is given by \( Q(u) \coloneqq F^{-1}(u) =
        \inf \{ x \in \mathbb{R} \mid F(x) \geq u \} \) for \( u \in [0, 1] \).

        \item The derivative of \(Q(u)\) w.r.t. \(u\) is called the \textit{quantile
        density function (qdf)}, and is denoted by \(q\). It is given by
        \(q(u) \coloneqq \frac{d}{du} Q(u)\) for \( u \in [0, 1] \).
    \end{enumerate}
\end{definition}
For clarity, we will use the arguments $x$ and $z$ for pdfs and cdfs, and the arguments
$u$ and $v$ for qfs and qdfs. Since we will later only look at densities from continuous
distributions on their support (i.e., where $f > 0$), we will in the following assume
that $F$ is strictly increasing and thus its inverse $Q$ to be well-defined.
The following relation between the pdf and qdf will be useful later when computing
transformations of the density functions \parencite[cf.][]{Jones1992}:
\begin{lemma}
\label{lemma:f eq inverse qdf}
    Let \(f\) be a pdf, \(F\) the corresponding cdf, \(Q\) the corresponding qf, and
    \(q\) the corresponding qdf. Then it holds that
    \begin{equation}
    \label{eq:qdfinversef}
        {q}(u) = \frac{1}{{f}({Q}(u))},
    \end{equation}
    and
    \begin{equation}
    \label{eq:finverseqdf}
        {f}(x) = \frac{1}{{q}({F}(x))},
    \end{equation}
\end{lemma}
\begin{proof}
    We will show that \eqref{eq:qdfinversef} holds, \eqref{eq:finverseqdf} follows
    analogously. Note that \( q = (F^{-1})' \), so we can use the inverse function
    rule to characterize \(q \) with the inverse $Q = F^{-1}$ and derivative $f = F'$
    of \( F \):
    \begin{equation}
    \label{eq:proof_qdfinversef}
        (F^{-1})'(u) = \frac{1}{{F'}({F^{-1}}(u))} = \frac{1}{{f}({Q}(u))}
    \end{equation}
\end{proof}
Because of the relation in \eqref{eq:qdfinversef}, the qdf is sometimes called the
"sparsity function" \parencite[cf.][]{Tukey1965}.
($\Omega$ denotes the space of random objects
we are interested in. $\mathcal{G}$ will denote the "set of distributions", i.e., the set containing
any class of functions that can represent a distribution.)

Following \textcite{PetersenZhangKokoszka2022}, we denote with
\begin{equation}
\label{eq:density_set}
    \mathcal{D} \subseteq \left\{ f : f(x) \geq 0, \int_{\mathbb{R}} f(x) \, dx = 1 \right\}
\end{equation}
the set of probability density functions (pdfs) that are of interest to us.\footnote{Note that
this in particular means that $\mathcal{D}$ can contain densities with different supports.
This will be significant later on when we perform transformations on these densities.}
$\Omega$ will be of use when generally explaining fda methods and theory, as well as
the concept of Fréchet regression. In the applications, we will be focussed on either
$\Omega = \mathcal{G}$ or $\Omega = \mathcal{D}$

The densities that are of interest to us are given in the following definition:
\begin{definition}
    Denote by $\mathcal{D}$ the space of continuous densities $f$, where each
    $f$ is supported on its own compact interval $[a_i, b_i]$ for some $a_i, b_i \in
    \mathbb{R}$ with $a_i < b_i$.
\end{definition}

\subsubsection{Stochastic Processes and Data Generation}
\label{sec:stochastic_processes}
\paragraph{Introduction to Stochastic Processes}
\label{sec:intro_stochastic}
\paragraph{Data Generation Mechanism}
\label{sec:data_generation}

\subsubsection{Fréchet Mean and Variance}
\label{sec:f_mean}
% Existing Content

\subsubsection{Wasserstein Distance and Its Properties}
\label{sec:wasserstein_distance}
% Existing Content

\subsubsection{Wasserstein-Fréchet Mean}
\label{sec:wasserstein_f_mean}
% Existing Content
d perhaps its relationship with the quantile synchronized mean.
