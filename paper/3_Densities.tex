(FIND PAPER WHERE I GOT THIS INTERPRETATION FROM!)
In this chapter we are going to examine the case of density regression, i.e., a
regression model with densities as responses. Since $\mathcal{D} \subseteq L^2(\mathbb{R})$,
one can choose to interpret the densities as elements of
the Hilbert space $\mathcal{L}^2$ and directly apply fda methods on the densities
\parencite[see e.g.][]{KneipUtikal2001}. However,
as $\mathcal{D}$ is not a linear subspace of $\mathcal{L}^2$, it is not guaranteed that
we yield results within $\mathcal{D}$ --- it only means we will yield results within $\mathcal{L}^2$.
Alternative characterizations of the space $\mathcal{D}$ have been explored,

This might at first glance seem like a
special case of functional regression since the space of densities on [a,b] is a subset
of L2([a,b]), so that fda methods are applicable to this problem. However, since the
space of densities is itself not a \textit{linear subspace} of L2, the results of the
fda methods

Distributions can be characterized by f, F, Q, and q, among others. Of these, the least
constraints are on q. These vanish if we
take the logarithm \parencite[cf.][]{KokoszkaEtAl2019}. Since our DGP generates densities on
different supports, we use \textcite{KokoszkaEtAl2019}s modification on the log quantile-density
transformation, i.e., our transformation yields a tuple ($psi(f), Q(s_0)$), with
$psi(f)$ being the transformation of f as defined by \textcite{PetersenMüller2016}, and
$Q(s_0)$ being the start of the density support ($s_0 > 0$ is fulfilled, since the support
of $Q$ is chosen away from zero for numerical reasons anyway). For the predicted $f_i$ hats,
in the inverse transformation the $s_0i$ from the corresponding observation $f_i$ was
assigned as estimate $s_0i$ hat.

For the descriptive Fréchet mean of the sample of densities, we had to estimate the
starting value for our modified inverse transformation method. Since it was a mean qdf,
we deemed it appropriate to take a mean starting value. This estimation step indroduces
uncertainty and thus our mean is not centered around zero, but close.

(Transformation approach regression: underestimating mu and sigma for lower x,
overestimating for higher x?) \textcite{PetersenLiuDivani2021}

\begin{figure}[h]
    \centering
    \input{../bld/figures/fda/broken_qdf.pgf}
    % \resizebox{0.9\textwidth}{!}{\input{../bld/figures/fda/broken_qdf.pgf}}
    \caption{A numerically broken qdf from a low-variance distribution. The values
    near the right boundary became too large to store for the computer.}
    \label{fig:broken_qdf}
\end{figure}

\begin{figure}[h]
    \centering
    \input{../bld/figures/fda/fixed_qdf.pgf}
    % \resizebox{0.9\textwidth}{!}{\input{../bld/figures/fda/fixed_qdf.pgf}}
    \caption{Fixed qdf by truncating the pdf support to where the pdf is greater than
    $\varepsilon = \nobreak 10^{-3}$.}
    \label{fig:fixed_qdf}
\end{figure}

\begin{figure}[h]
    \centering
    \input{../bld/figures/fda/naive_trunc_rep_vs_orig.pgf}
    % \resizebox{0.9\textwidth}{!}{\input{../bld/figures/fda/naive_trunc_rep_vs_orig.pgf}}
    \caption{Naive truncated Karhunen-Loève representation of a sample density. Added
    small perturbation to differentiate the curves for $K=1$ and $K=2$.}
    \label{fig:naive_trunc_rep}
\end{figure}

\begin{figure}[h]
    \centering
    \input{../bld/figures/fda/trunc_rep_vs_orig.pgf}
    % \resizebox{0.9\textwidth}{!}{\input{../bld/figures/fda/trunc_rep_vs_orig.pgf}}
    \caption{Truncated Karhunen-Loève representation of a sample density.}
    \label{fig:trunc_rep}
\end{figure}

\begin{figure}[h]
    \centering
    \input{../bld/figures/fda/1st_modes.pgf}
    % \resizebox{0.9\textwidth}{!}{\input{../bld/figures/fda/1st_modes.pgf}}
    \caption{Comparison of first mode of variation of naive approach and of LQD approach.}
    \label{fig:1st_modes}
\end{figure}

\begin{figure}[h]
    \centering
    \input{../bld/figures/fda/2nd_modes.pgf}
    % \resizebox{0.9\textwidth}{!}{\input{../bld/figures/fda/2nd_modes.pgf}}
    \caption{Comparison of second mode of variation of naive approach and of LQD approach.}
    \label{fig:2nd_modes}
\end{figure}

\begin{figure}[h]
    \centering
    \input{../bld/figures/fda/frechet_means.pgf}
    % \resizebox{0.9\textwidth}{!}{\input{../bld/figures/fda/frechet_means.pgf}}
    \caption{Simulation results for calculating Fréchet mean densities with different
    sample sizes.}
    \label{fig:sim_f_mean}
\end{figure}

\begin{figure}[h]
    \centering
    \input{../bld/figures/fda/f_mean_vs_denstimation.pgf}
    % \resizebox{0.9\textwidth}{!}{\input{../bld/figures/fda/frechet_means.pgf}}
    \caption{Simulation results of Fréchet mean of observed densities vs.
    Fréchet mean of estimated densities.}
    \label{fig:sim_f_denstimation}
\end{figure}

\begin{figure}[h]
    \centering
    \input{../bld/figures/fda/cs_mean_vs_denstimation.pgf}
    % \resizebox{0.9\textwidth}{!}{\input{../bld/figures/fda/frechet_means.pgf}}
    \caption{Simulation results of cross-sectional mean of observed densities vs.
    cross-sectional mean of estimated densities.}
    \label{fig:sim_cs_denstimation}
\end{figure}

\begin{figure}[h]
    \centering
    \input{../bld/figures/fda/comparison_f_cs.pgf}
    % \resizebox{0.9\textwidth}{!}{\input{../bld/figures/fda/frechet_means.pgf}}
    \caption{Simulation results of Fréchet mean plotted against "naive" cross-sectional mean for
    $n = 200$.}
    \label{fig:sim_f_vs_cs}
\end{figure}

Optimal K for directly observed densities always 1, so graphic omitted.

\begin{figure}[h]
    \centering
    \input{../bld/figures/fda/k_opt_denstimation_histogram.pgf}
    % \resizebox{0.9\textwidth}{!}{\input{../bld/figures/fda/frechet_means.pgf}}
    \caption{Simulation results for optimal choice of $K$ in context of density estimation,
    given that $\text{fve} \geq 90 \%$}
    \label{fig:sim_k_opt_denstimation}
\end{figure}

\begin{figure}[h]
    \centering
    \input{../bld/figures/fda/fve.pgf}
    % \resizebox{0.9\textwidth}{!}{\input{../bld/figures/fda/frechet_means.pgf}}
    \caption{Boxplot of Fréchet fraction of variance explained with directly observed
    densities, when $K = 1$.}
    \label{fig:sim_fve}
\end{figure}

\begin{figure}[h]
    \centering
    \input{../bld/figures/fda/fve_denstimation.pgf}
    % \resizebox{0.9\textwidth}{!}{\input{../bld/figures/fda/frechet_means.pgf}}
    \caption{Boxplot of Fréchet fraction of variance explained with previously estimated
    densities, when $K = 1$.}
    \label{fig:sim_fve_denstimation}
\end{figure}

For densities with smaller sigmas, we get very low density values for big parts of the
support, leading to numerical artifacts when calculating the quantile densities (infs
to the right, astronomical values to the left) which breaks further calculations and
makes it impossible to get the original densities from those broken qds again. We chose
to calculate the "effective" support of the densities, if they have very small sigmas,
defined by the density values being larger than some epsilon on this support. This
makes the transformation method from \textcite{PetersenMüller2016} inadequate, hence we
used the modified transformation from \textcite{KokoszkaEtAl2019}.\footnote{In their own R
package (\citetitle{fdadensity}), the authors have "RegularizeByAlpha" function, which tries to raise the minimal
value of a density to a given level alpha (and normalizes afterward, so we still have a
valid density). This also gets rid of the numerical artifacts and doesnt change the
support of the function, which makes the methods from their paper still adequate.}

For the prediction of the start values, we linearly interpolated between the given start
values from the predictor observations. The ISE increases with sample size, which must
be explained by the uncertainty introduced through having to estimate the values in the
inverse transformation.

Moreover, the quality of the density estimates deteriorates with sample size, suggesting
to vary the bandwidth choice with sample size, as \textcite{PetersenMüller2019} did in
their comparison of the Fréchet estimator with the Nadaraya-Watson estimator. Due to
time constraints and the computational requirements, we did not conduct such
hyperparameter tuning but choose a rule-of-thumb bandwidth
\parencites[Chapter~3.4.1]{Silverman1986}[Chapter~2.2.1]{LiRacine2007}
