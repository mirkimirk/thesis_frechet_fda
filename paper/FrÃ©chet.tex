This section will give an introduction to the concept of Fréchet regression
and some background.

\subsection{Conditional Fréchet Mean and Variance}
\label{sec:condmean}

\subsection{General model}
\label{sec:general}

\begin{figure}[h]
    \centering
    % \input{../bld/figures/frechet/1stlastqf.pgf}
    \resizebox{1\textwidth}{!}{\input{../bld/figures/frechet/1stlastqf.pgf}}
    \caption[Two sample qfs]{Plot of two sample qfs, one with predictor value $p = -1$,
    and the other for $p = 1$.}
    \label{fig:1stlastqf}
\end{figure}

Figure \ref{fig:1stlastqf} shows two example qfs, generated from the above mentioned
process with the predictor values $p=-1$ and $p=1$. One can see that both location and
scale are much larger for the curve with the higher predictor value.

\begin{figure}[h]
    \centering
    % \input{../bld/figures/frechet/some_densities.pgf}
    \resizebox{1\textwidth}{!}{\input{../bld/figures/frechet/some_densities.pgf}}
    \caption[Several sampled pdfs]{Plot of several sample pdfs.}
    \label{fig:some_densities}
\end{figure}

Figure \ref{fig:some_densities} plots the corresponding pdfs of some of the generated qfs.

\begin{figure}[h]
    \centering
    % \input{../bld/figures/frechet/frechet_estimates.pgf}
    \resizebox{1\textwidth}{!}{\input{../bld/figures/frechet/frechet_estimates.pgf}}
    \caption[Estimated pdfs from Fréchet method]{Plot of estimated densities via Fréchet regression method.
    The densities from left to right are corresponding to a predictor value $p$ on an
    equidistant grid on $[-1,1]$ with 50 grid points.}
    \label{fig:frechet_estimates}
\end{figure}

\begin{figure}[h]
    \centering
    % \input{../bld/figures/frechet/frechet_estimates_3d.pgf}
    \resizebox{1\textwidth}{!}{\input{../bld/figures/frechet/frechet_estimates_3d.pgf}}
    \caption[Estimated qfs from Fréchet method]{Plot of estimated qfs via Fréchet regression method.}
    \label{fig:frechet_estimates_3d}
\end{figure}

Figures \ref{fig:frechet_estimates} and \ref{fig:frechet_estimates_3d} plot the Fréchet
estimates of the pdfs and qfs respectively. In the first figure, the estimated pdfs for
different predictor values are drawn on top of each other, and in the second, a 3d plot
was used to visualize the change of shape of the qfs with a One can see a smooth
increase in location and variance in both figures with an increase in the predictor
value.

% \begin{figure}[h]
%     \centering
%     % \input{../bld/figures/frechet/all_observations_3d.pgf}
%     \resizebox{1\textwidth}{!}{\input{../bld/figures/frechet/all_observations_3d.pgf}}
%     \caption{Plot of sampled qfs and their predictor values.}
%     \label{fig:all_observations_3d}
% \end{figure}

\begin{figure}[h]
    \centering
    % \input{../bld/figures/frechet/frechet_est_vs_true.pgf}
    \resizebox{1\textwidth}{!}{\input{../bld/figures/frechet/frechet_est_vs_true.pgf}}
    \caption[Comparison: estimated vs. true densities --- Fréchet]{Estimated
    densities from Fréchet regression plotted against the true densities for some selected
    predictor values.}
    \label{fig:frechet_est_vs_true}
\end{figure}

It can be shown that $d_Q(f,g)$ = $d_W(f,g)$, for any two densities $f$ and $g$ in the
space. (Also explain how quantile synchronized mean different and better than naive
cross-sectional mean.)
\begin{lemma}
    \label{lemma:dqeqdw}
    $d_Q = d_W$
\end{lemma}
\begin{proof}
    asdsadasdads
\end{proof}

Then we can see that:
\begin{lemma}
    The Wasserstein-Fréchet mean estimator is given by
    \begin{equation}
        \hat{f}_\oplus(x) = \frac{1}{\hat{q}_\oplus(\hat{F}_\oplus(x))},
    \end{equation}
    with $\hat{q}_\oplus = \frac{1}{n} \sum_{1}^{n} q_i$.
\end{lemma}
\begin{proof}
    By \ref{lemma:dqeqdw}, we can substitute the quantile distance for the
    Wasserstein distance in the computation of the Fréchet mean. We set
    $Q_\oplus(t) = E[Q(t)]$ because $E[Q(t)]$ is the minimizer of
    \begin{equation}
    \label{eq:wf_mean}
        \begin{aligned}
            E[d_w^2(f_i, f_\oplus)]	& =
            E\left[\int_{0}^{1}(F_i^{-1}(t) - F_\oplus^{-1}(t))^2 \,dt\right] \\
                                    & =
            E\left[\int_{0}^{1}(Q_i(t) - Q_\oplus(t))^2 \,dt\right],
        \end{aligned}
    \end{equation}
    see \citet[Chapter~3.1.4]{PanaretosZemel2020}. To compute the corresponding density
    function, use inverse function rule to get
    $f_\oplus(x) = \frac{1}{q_\oplus(F_\oplus(x))}$, which shows that it suffices to
    estimate $q_\oplus$. Remember that
    $q_\oplus = \frac{\mathrm{d}Q_\oplus}{\mathrm{d}t}$. Because of Assumption REFER A1,
    we can pass the differentiation inside the expectation to see
    $q_\oplus = E\left[\frac{\mathrm{d}Q}{\mathrm{d}t}\right]$, which by analogy
    principle suggests to average the sample observed (or previously estimated)
    quantile densities $q_i$ to obtain an estimator $\hat{q}_\oplus$ for $q_\oplus$.
\end{proof}


Calculation of quadratic programming problem done according to algorithm in \textcite{PetersenLiuDivani2021}.
Checking for estimated qf as valid solution saved much computing time.