This section will give an introduction to the concept of Fréchet regression
and some background.

\subsection{Conditional Fréchet Mean}
\label{sec:cond_fmean}
The authors extend the notion of Fréchet mean given in Section~\ref{sec:f_mean} to that
of a conditional Fréchet mean, with a corresponding conditional Fréchet variance:
\begin{equation}
    m_\oplus(x) = \argmin_{\omega \in \Omega} M_\oplus(\omega, x)
\end{equation}

\begin{equation}
    M_\oplus(\cdot, x) = E \left[ d^2(Y, \cdot) \mid X = x \right]
\end{equation}
This is done characterizing the well-known linear regression model in terms

\subsection{Wasserstein Space}
\label{sec:wasserstein_space}

\subsection{Computational Details}
\label{sec:computation}

CHECK THIS ALGORITHM FOR NOTATION
\begin{algorithm}
    \caption{Estimating \(\hat{Q}(x)\)}
    \label{alg:quadprog}
    \begin{algorithmic}[1]
    \Require Predictor vector \( x \in \mathbb{R}^p \); quantile functions \( Q_i \), and grid \( 0 = t_1 < \ldots < t_m = 1 \)
    \Ensure Estimates \( \hat{Q}(x; t_l); l = 1, \ldots, m \)
    \For{\( l = 1, \ldots, m \)}
        \State Compute \( Q_l = \frac{1}{n} \sum_{i=1}^{n} \sin(x) Q_i(t_l) \)
    \EndFor
    \If{\( Q_{l+1} \geq Q_l \) for all \( l \in \{1, \ldots, m-1\} \)}
        \State Set \( \hat{Q}(x; t_l) = Q_l \)
    \Else
        \State Compute \( b^* = \min_{b \in \mathbb{R}^m} \frac{1}{2} b^\top A b - Q^\top A b \) subject to \( b_1 \leq \ldots \leq b_m \)
        \State Set \( \hat{Q}(x; t_l) = b^*_l \)
    \EndIf
    \end{algorithmic}
\end{algorithm}


Calculation of quadratic programming problem done according to algorithm in \textcite{PetersenLiuDivani2021}.
Checking for estimated qf as valid solution saved much computing time.
