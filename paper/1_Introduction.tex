This thesis is about comparing different methods to solving the task of density
regression. In functional data analysis (FDA) literature, several modifications to
the usual methods to examine functional data have been proposed to handle the case of
density functions as responses in a regression setting. As the space of density functions is itself
not a linear space, key assumptions from FDA literature are violated and the results
of the usual procedures are not guaranteed to stay in the density space.

We will focus on the transformation method described in \textcite{PetersenMüller2016}.
This approach guarantees to get results in the density space when applying the usual
FDA methods on a transformed space and mapping the transformed values back via an
inverse mapping into the density space.

A different strand of literature is the one about object-oriented data analysis. To this,
functional data is a special case of a general random object. \textcite{PetersenMüller2019}
explored a method for a general form of regression where the response is only assumed to
be metric space-valued. As this is a general approach encompassing the case of the
response space being the space of densities (with a suitable metric defined on it),
this method should be more flexible in adjusting to the structure of the space and
provide a more direct alternative to the transformation method and other approaches
trying to make FDA methods work.

In the following we will compare both methods in a simulation study, both in a case
where the sample of densities is directly observed an in the more realistic case of
first having to obtain the sample via density estimation. Before, section 2 will give
a review of the mathematical requisites for FDA and the Fréchet regression framework.
Section 3 will explain the setup and results of the simulation study, and chapter 4
concludes.
