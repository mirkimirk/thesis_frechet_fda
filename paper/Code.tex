In this chapter, some code snippets are provided for the most important functions we
used.

Central in our code is the class \textit{Function} that we defined. It can hold an analytical
function, with atributes \textit{x} (containing the domain values), \textit{y} (containing the function
values), and \textit{grid\_size} (the number of discretization points at which the function
is evaluated), among others. It has many different methods, like \textit{integrate()},
\textit{differentiate()}, \textit{invert()} etc., that are used throughout the code and yield new 
Function class objects. It can be inspected in the "function\_class.py" file in the
GitHub repository (\url{https://github.com/mirkimirk/thesis_frechet_fda}) and would be
too large to print here.

\subsection{Python Code for pdf to qdf Conversion}
\label{sec:code_pdf_to_qdf}
The following code shows how we converted pdfs to qdfs and vice versa.

The option \textit{center\_on\_zero} in the \textit{qdf\_to\_pdf} function artificially
centers the resulting pdf around zero. It is not used in any of the plots in this thesis.

\begin{lstlisting}[style=python]
    def pdf_to_qdf(pdf: Function, save_support_start: bool = False) -> Function:
        """Directly convert a pdf to a qdf using inverse function rule on qf."""
        quantile_func = pdf.integrate().invert()
        if save_support_start:
            return (1 / pdf.compose(quantile_func), quantile_func.y[0])
        else:
            return 1 / pdf.compose(quantile_func)
    
    def qdf_to_pdf(
        qdf: Function, start_val: float = 0, center_on_zero: bool = False,
    ) -> Function:
        """Directly convert a qdf to a pdf using inverse function rule on cdf."""
        if center_on_zero:
            cdf = qdf.integrate().vcenter().invert()
        else:
            cdf = (qdf.integrate() + start_val).invert()
        return 1 / qdf.compose(cdf)
\end{lstlisting}

\subsection{Optimal Range Calculation in Python}
\label{sec:code_optimal_range}
This code shows our computation of the "effective" range by narrowing the domain
of the pdf to the points where it is larger than $\delta = 10^{-3}$
\begin{lstlisting}[style=python]
    def get_optimal_range(funcs: list[Function], delta: float = 1e-3) -> np.ndarray:
        """Get narrower support if density values are too small (smaller than delta).

        This is used so the qdfs dont get astronomically large at the boundaries and destroy
        numerical methods.

        Note: The method here assumes that the functions do have a compact
        support (even if it is narrower than the initial support). So if there is a point x1
        where func.y > delta is true, another point x2 > x1 where it is not, and then
        another x3 > x2 where it is true again, then x2 is included in the new range
        although it does not fullfill the condition.

        """
        new_ranges = np.zeros((len(funcs), 2))
        for i, func in enumerate(funcs):
            support_to_keep = func.x[func.y > delta]
            new_ranges[i] = (support_to_keep[0], support_to_keep[-1])
        return new_ranges
\end{lstlisting}

\newpage
\subsection{LQD Transformations in Python}
\label{sec:code_lqd}
The following code shows our Python implementation of the LQD transformation and its
inverse.
\begin{lstlisting}[style=python]
    def log_qd_transform(
        densities_sample: list[Function], different_supports: bool = False,
    ) -> list[Function]:
        """Perform log quantile density transformation on a density sample."""
        qdfs = [
            pdf_to_qdf(density.drop_inf(), different_supports)
            for density in densities_sample
        ]
        if different_supports:
            qdfs_and_start_vals = np.array(qdfs)
            lqdfs = [qdf.log() for qdf in qdfs_and_start_vals[:, 0]]
            start_vals = np.array(qdfs_and_start_vals[:, 1], dtype=np.float64)
            return lqdfs, start_vals
        else:
            return [qdf.log() for qdf in qdfs]

    def inverse_log_qd_transform(
        transformed_funcs: list[Function], start_of_support: list[float] = None,
    ) -> list[Function]:
        """Transform back into density space."""
        natural_qfs = [func.exp().integrate() for func in transformed_funcs]
        if start_of_support is not None:
            natural_qfs += start_of_support
            cdfs = [qf.invert() for qf in natural_qfs]
        else:
            cdfs = [qf.vcenter().invert() for qf in natural_qfs]
        exponents = [
            -func.compose(cdf) for func, cdf in zip(transformed_funcs, cdfs, strict=True)
        ]
        return [exponent.exp() for exponent in exponents]
\end{lstlisting}

\newpage
\subsection{Solving Fréchet Quadratic Programming Problem}
\label{sec:code_quadratic_program}
The following Python code shows the implementation of the quadratic programming problem
to solve for the Fréchet means.
\begin{lstlisting}[style=Python]
    def solve_frechet_qp(
        xs_to_predict: np.ndarray,
        x_observed: np.ndarray,
        quantile_functions: list[Function],
    ) -> list[Function]:
        """Sets up quadratic programming problem and solves it."""
        estimates = []
        for x in xs_to_predict:
            # Estimate condtional qf, drop support where values become nan or inf
            estimated_qf = qf_tilde(x, x_observed, quantile_functions).drop_inf()
            constraints_check = estimated_qf.y[1:] - estimated_qf.y[:-1]
            if np.all(constraints_check > 0):
                # If estimator valid qf, it is the optimal solution
                estimates.append(estimated_qf)
            else:
                # Else, find closest vector to estimator that is a valid solution
                qp_a = estimated_qf.y
                grid_size = len(qp_a)
                qp_g = np.identity(grid_size)
                qp_c = np.eye(grid_size, grid_size - 1, k=-1) - np.eye(
                    grid_size, grid_size - 1,
                )
                qp_b = np.zeros(grid_size - 1)
                solution = quadprog.solve_qp(qp_g, qp_a, qp_c, qp_b)[0]
                estimates.append(Function(estimated_qf.x, solution))
        return estimates
\end{lstlisting}
