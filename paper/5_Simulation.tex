We see a more continous estimate in the Fréchet regression. Clearly the mean shifts
steadily to the right, as x increases, as well as more variance.

Because in the functional regression on the log transformed qds, we have to adjust with
the range with the individual quantile of the observation, and otherwise there is no
possibility to keep the mean, the linear dependence of the mean is lost. One can though
see that the variance rises with x. This information is not lost in the log qdfs.

\begin{table}[h]
    \centering
    \begin{tabular}{lccc}
        \hline
        Method & Value 1 & Value 2 & Value 3 \\
        \hline
        Mean\_ISE\_Frechet & 0.18 & 0.08 & 0.04 \\
        Mean\_ISE\_Func\_Reg & 34.49 & 31.19 & 31.88 \\
        Mean\_ISE\_Frechet\_Denstimation & 32.31 & 27.98 & 27.38 \\
        Mean\_ISE\_Func\_Reg\_Denstimation & 47.08 & 43.39 & 43.43 \\
        \hline
    \end{tabular}
    \caption{Mean ISE Values}
\end{table}


The curve for the Fréchet mean with observed densities is farther off than the Fréchet
mean with a previous density estimation step. We think this must be because the densities
yielded by the density estimation have a smaller variance in the left bounds of their
support, so that the uncertainty introduced by the estimation of the new left bound for
the average is smaller.