Notably during the computation of the Wasserstein-Fréchet mean of a sample of density
functions, we encountered numerical problems. Since the computation of the mean involves
obtaining the quantile densities from the densities, the densities from the small
variance distributions are very close to zero for much of their support, leading to the
qfs to have a very steep slope (i.e. high valued qdfs) near the boundaries. These
astronomical values (even "infs") lead to problems when computing the mean qdf and then
transforming this to a corresponding pdf.

\begin{figure}[h]
    \centering
    % \input{../bld/figures/fda/broken_qdf.pgf}
    \resizebox{0.9\textwidth}{!}{\input{../bld/figures/fda/broken_qdf.pgf}}
    \caption[Example of broken qdf]{A numerically broken qdf from a low-variance distribution.
    The values near the right boundary became too large to store for the computer.}
    \label{fig:broken_qdf}
\end{figure}

\begin{figure}[h]
    \centering
    % \input{../bld/figures/fda/fixed_qdf.pgf}
    \resizebox{0.9\textwidth}{!}{\input{../bld/figures/fda/fixed_qdf.pgf}}
    \caption[Example of fixed qdf]{Fixed qdf by truncating the pdf support to where
    the pdf is greater than $\varepsilon = \nobreak 10^{-3}$.}
    \label{fig:fixed_qdf}
\end{figure}

Figure \ref{fig:broken_qdf} shows one of the broken qdfs, and Figure \ref{fig:fixed_qdf}
shows the fixed version of it by truncating the support of the pdf:

In their own R package (\citetitle{fdadensity}), the authors have "RegularizeByAlpha"
function, which tries to raise the minimal value of a density to a given level alpha. This
also gets rid of the numerical artifacts and doesnt change the support of the function,
which makes the methods from their paper still adequate. At the end of calculations, they
revert this change with "DeRegularizeByAlpha".