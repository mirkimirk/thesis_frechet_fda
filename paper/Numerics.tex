For densities with smaller sigmas, we get very low density values for large parts of the
support, leading to numerical artifacts when calculating the quantile densities (infs
near the right bound, astronomical values near the left bound) since the qfs corresponding to
those low variance pdfs have a very steep slope near the boundaries. These artifacts
break further calculations and make it impossible to get the original densities from those broken qds
again. We chose to calculate the "effective" support of the densities, if they have very
small sigmas, defined by the density values being larger than some small $\delta > 0$ on
this support (see Appendix~\ref{sec:code_optimal_range}). This makes the transformation
method from \textcite{PetersenMüller2016} inadequate (which assumes a common support),
hence we used the modified transformation from \textcite{KokoszkaEtAl2019}.

\begin{figure}[h]
    \centering
    % \input{../bld/figures/fda/broken_qdf.pgf}
    \resizebox{0.9\textwidth}{!}{\input{../bld/figures/fda/broken_qdf.pgf}}
    \caption[Example of broken qdf]{A numerically broken qdf from a low-variance distribution.
    The values near the right boundary became too large to store for the computer.}
    \label{fig:broken_qdf}
\end{figure}

\begin{figure}[h]
    \centering
    % \input{../bld/figures/fda/fixed_qdf.pgf}
    \resizebox{0.9\textwidth}{!}{\input{../bld/figures/fda/fixed_qdf.pgf}}
    \caption[Example of fixed qdf]{Fixed qdf by truncating the pdf support to where
    the pdf is greater than $\varepsilon = \nobreak 10^{-3}$.}
    \label{fig:fixed_qdf}
\end{figure}

Figure \ref{fig:broken_qdf} shows one of the broken qdfs, and Figure \ref{fig:fixed_qdf}
shows the fixed version of it by truncating the support of the pdf:

In their own R package (\citetitle{fdadensity}), the authors have a "RegularizeByAlpha"
function, which tries to raise the minimal value of a density to a given level alpha. This
also gets rid of the numerical artifacts and does not change the support of the function,
which makes the methods from their paper still adequate. At the end of the calculations, they
revert this change with "DeRegularizeByAlpha".