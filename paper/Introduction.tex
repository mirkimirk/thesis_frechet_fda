This thesis undertakes the task of comparing different approaches to 
the problem of density regression. Specifically, we will juxtapose the 
novel Fréchet regression method against a particular method from Functional Data 
Analysis (fda) literature. Within the scope of this thesis, the term Fréchet 
regression is confined to the global regression context.

In the fda literature, various methods have been adapted to handle density 
functions as responses within a regression framework. Density data have found 
applications in a multitude of fields, such as income distributions 
\parencite{HildenbrandKneipUtikal1999,KneipUtikal2001} and financial markets \parencite{KokoszkaEtAl2019}. 
However, the space of density functions is inherently not a linear space, thereby 
violating key assumptions in fda. This non-linearity makes it challenging to 
ensure that the outcomes of standard procedures remain in the density space.

Our primary focus is the transformation method as delineated by 
\textcite{PetersenMüller2016}, which employs a log quantile density (LQD) 
transformation. This approach has been slightly modified by \textcite{KokoszkaEtAl2019}
and ensures that the results stay within the density space by performing standard 
fda methods in a transformed space and then inverting this transformation. A 
comprehensive overview of fda can be found in \textcite{WangChiouMüller2016}, 
and an in-depth discussion on the modeling of density functions is presented in 
\textcite{PetersenZhangKokoszka2022}.

In contrast, another emerging avenue of research is object-oriented data analysis, 
which offers a more generalized perspective on functional data. Here, functional 
data serve as a special case within a broader category of random objects. 
\textcite{PetersenMüller2019} have delved into this approach and proposed a 
regression method where the response variable is assumed to be in a metric space. 
This flexibility makes it a more direct and adaptable alternative to the 
transformation methods and other adaptations of fda, especially since it doesn't 
rely on the "extrinsic" properties of the Hilbert space of which the density space 
is a subset.

To empirically compare these two frameworks, a simulation study will be conducted. 
This study will assess the efficacy of each method in scenarios where density samples 
are directly observed and in more realistic settings where these samples are estimated. 
The remainder of this thesis is organized as follows: Section 2 offers a mathematical 
foundation for fda and the Fréchet regression framework, Section 3 explain basic fda
concepts. Section 4 discusses density regression and the inadequacy of directly applying
fda methods on the space of densities, along with the transfromation approach. Section 5
explain the Fréchet method, with a special focus on our regression context in the Wasserstein
space of distributions. Section 6 illustrates the methods, and compares them in a simulation study.
Section 7 concludes.

The code to this thesis can be found in \url{https://github.com/mirkimirk/thesis_frechet_fda}log quantile density (LQD)
