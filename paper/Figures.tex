This chapter contains additional figures.

\subsection{Illustration: Naive approach vs LQD approach}
\label{sec:illustration_lqd}

\subsubsection{Data}
\label{sec:dgp_descriptives}
We generated data as in the first simulation scenario in \textcite{PetersenMüller2016}:
We sample 200 densities from truncated normal distributions, with $\mu = 0$ in the
interval $[-3,3]$. The sigmas are sampled as follows: we sample $\log(\sigma) \sim U(-1,1)$
200 times, and then obtain $\sigma$ by calculating $\exp(\log(\sigma))$. Thus we have a sample
$f_1, \nobreak \dots, \nobreak f_{200}$ of densities, with $f_i$ being some truncated
normal distribution with varying variances, centered on zero. The true center of this DGP
is a truncated standard normal distribution, so thats what we would expect to see as
estimated mean.

As a measure of explained variance, we used the Fréchet variance explained (fve) that
\textcite{PetersenMüller2016} defined. Using the Fréchet variance $V_\oplus$ defined
in~\ref{eq:fvar}.

\begin{equation}
    \tilde{V}_\infty = \frac{1}{n} \sum_{i=1}^{n} d_W(f_i, \tilde{f}_\oplus)^2
    \label{eq:tilde_V_infinity}
\end{equation}

\begin{equation}
    V_K := V_\infty - \mathbb{E}\left[d_W(f_i, f_{i,K})^2\right]
    \label{eq:V_K}
\end{equation}

\begin{equation}
    \tilde{V}_K = \tilde{V}_\infty - \frac{1}{n} \sum_{i=1}^{n} d_W(f_i, \tilde{f}_{i,K})^2
    \label{eq:tilde_V_K}
\end{equation}

\begin{equation}
    \text{FVE} = \frac{V_K}{V_\infty}
    \label{eq:FVE}
\end{equation}

\begin{equation}
    \tilde{\text{FVE}} = \frac{\tilde{V}_K}{\tilde{V}_\infty}
    \label{eq:tilde_FVE}
\end{equation}

\begin{equation}
    K^* = \min \left\{ K : \frac{V_K}{V_\infty} > p \right\}
    \label{eq:K_star}
\end{equation}

\begin{equation}
    \tilde{K}^* = \min \left\{ K : \frac{\tilde{V}_K}{\tilde{V}_\infty} > p \right\}
    \label{eq:tilde_K_star}
\end{equation}


We used this measure in two ways: frist, to judge the number $K$ of how many components need to
be included in the Karhunen-Loève representation to reach a given level of variance explained;
and second, to compare the accuracy of the LQD method in the scenario with additional density estimation
step to the one without this step. For this, $K = 1$ was fixed as this is the true number
of principal components driving the variation in the sample curves (all variation in the sample is given
by the variation in one parameter $\sigma$). Then the fve was calculated on the basis
of the resultant Karhunen-Loève representations.

Before we present the simulation results, we turn to another visual comparison of
the naive application of fda methods with the LQD method.

\subsubsection{Modes of Variation}
\label{sec:mov_example}

\begin{figure}[h]
    \centering
    % \input{../bld/figures/fda/1st_modes.pgf}
    \resizebox{0.9\textwidth}{!}{\input{../bld/figures/fda/1st_modes.pgf}}
    \caption[Comparison: first mode of variation]{Comparison of first mode of variation
    of naive approach and of LQD approach.}
    \label{fig:1st_modes}
\end{figure}

\begin{figure}[h]
    \centering
    % \input{../bld/figures/fda/2nd_modes.pgf}
    \resizebox{0.9\textwidth}{!}{\input{../bld/figures/fda/2nd_modes.pgf}}
    \caption[Comparison: second mode of variation]{Comparison of second mode of variation
    of naive approach and of LQD approach.}
    \label{fig:2nd_modes}
\end{figure}

\subsubsection{Simulation results}
\label{sec:illustration_simulation}
The density estimation procedure is the same as explained in Section~\ref{sec:app_density_estimation}
but with one deviation: here, we used the boundary corrected estimator defined in
\textcite{PetersenMüller2016}. Since we used the standard normal kernel, the estimator
as defined in this paper, was not continuous. So we made it continuous by setting the
weight function equal to
\begin{equation*}
    \left( \int_{-1}^{1} \kappa(u) du \right)^{-1}
\end{equation*}

\begin{figure}[h]
    \centering
    \input{../bld/figures/fda/f_mean_vs_denstimation.pgf}
    % \resizebox{0.9\textwidth}{!}{\input{../bld/figures/fda/f_mean_vs_denstimation.pgf}}
    \caption[Simulation results: observed vs estimated densities --- Fréchet mean]{Simulation
    results of Fréchet mean of observed densities vs. Fréchet mean of estimated densities,
    $n=200$.}
    \label{fig:sim_f_denstimation}
\end{figure}

\begin{figure}[h]
    \centering
    % \input{../bld/figures/fda/cs_mean_vs_denstimation.pgf}
    \resizebox{0.9\textwidth}{!}{\input{../bld/figures/fda/cs_mean_vs_denstimation.pgf}}
    \caption[Simulation results: observed vs estimated densities --- cross-sectional mean]{Simulation
    results of cross-sectional mean of observed densities vs. cross-sectional mean of
    estimated densities, $n=200$.}
    \label{fig:sim_cs_denstimation}
\end{figure}

\begin{figure}[h]
    \centering
    % \input{../bld/figures/fda/comparison_f_cs.pgf}
    \resizebox{0.9\textwidth}{!}{\input{../bld/figures/fda/comparison_f_cs.pgf}}
    \caption[Simulation results: Fréchet mean vs cross-sectional mean]{Simulation
    results of Fréchet mean plotted against "naive" cross-sectional mean for $n = 200$.}
    \label{fig:sim_f_vs_cs}
\end{figure}

\begin{figure}[h]
    \centering
    % \input{../bld/figures/fda/frechet_means.pgf}
    \resizebox{0.9\textwidth}{!}{\input{../bld/figures/fda/frechet_means.pgf}}
    \caption[Simulation results: average Fréchet means]{Simulation results for calculating
    Fréchet mean densities with different sample sizes.}
    \label{fig:sim_f_mean}
\end{figure}

\begin{figure}[h]
    \centering
    % \input{../bld/figures/fda/k_opt_denstimation_histogram.pgf}
    \resizebox{0.9\textwidth}{!}{\input{../bld/figures/fda/k_opt_denstimation_histogram.pgf}}
    \caption[Simulation results: optimal $K$]{Simulation results for optimal choice of
    $K$ in context of density estimation, given that $\text{fve} \geq 90 \%$}
    \label{fig:sim_k_opt_denstimation}
\end{figure}

(Optimal K for directly observed densities always 1, so graphic omitted.)

\begin{figure}[h]
    \centering
    % \input{../bld/figures/fda/fve.pgf}
    \resizebox{0.9\textwidth}{!}{\input{../bld/figures/fda/fve.pgf}}
    \caption[Simulation results: boxplots fve --- observed densities]{Boxplot of
    Fréchet fraction of variance explained with directly observed
    densities, when $K = 1$.}
    \label{fig:sim_fve}
\end{figure}

\begin{figure}[h]
    \centering
    % \input{../bld/figures/fda/fve_denstimation.pgf}
    \resizebox{0.9\textwidth}{!}{\input{../bld/figures/fda/fve_denstimation.pgf}}
    \caption[Simulation results: boxplots fve --- estimated densities]{Boxplot of
    Fréchet fraction of variance explained with previously estimated densities, when
    $K = 1$.}
    \label{fig:sim_fve_denstimation}
\end{figure}



















%%%% NOT NEEDED; WE PLOT THEM SMALLER IN THE MAIN TEXT
% \begin{figure}[h]
%     \centering
%     % \input{../bld/figures/fda/naive_trunc_rep_vs_orig.pgf}
%     \resizebox{0.9\textwidth}{!}{\input{../bld/figures/fda/naive_trunc_rep_vs_orig.pgf}}
%     \caption[Truncated representation --- naive]{Naive truncated Karhunen-Loève
%     representation of a sample density. Added small perturbation to differentiate the
%     curves for $K=1$ and $K=2$.}
%     \label{fig:naive_trunc_rep}
% \end{figure}

% \begin{figure}[h]
%     \centering
%     % \input{../bld/figures/fda/trunc_rep_vs_orig.pgf}
%     \resizebox{0.9\textwidth}{!}{\input{../bld/figures/fda/trunc_rep_vs_orig.pgf}}
%     \caption[Truncated representation --- LQD method]{Truncated Karhunen-Loève
%     representation of a sample density with LQD method.}
%     \label{fig:trunc_rep}
% \end{figure}
