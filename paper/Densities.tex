In this chapter we are going to examine the case of density regression, i.e., a
regression model with densities as responses.

\subsection{Approaches}
\label{sec:approaches}
Here, we will briefly outline different interpretations of the density space, and the
approaches that follow, to motivate our choice of the transformation method from
\textcites{PetersenMüller2016}{KokoszkaEtAl2019}.

\subsubsection{Subset of Hilbert Space}
\label{sec:l2_interpretation}
If one additionally assumes that $\mathcal{D} \subseteq L^2(\mathbb{R})$, one can choose
to interpret the densities as elements of the Hilbert space $\mathcal{L}^2$ and directly
apply fda methods on the densities \parencite[see e.g.][]{KneipUtikal2001}. Then one
could handle the regression problem by applying the functional response regression method
first described by \textcite{Faraway1997}.

However, as $\mathcal{D}$ is not a linear subspace of $\mathcal{L}^2$, it is not guaranteed that
we yield results within $\mathcal{D}$ --- it only means we will yield results within $\mathcal{L}^2$,
so our estimates, modes of variation etc. are not guaranteed to be themselves densities.
Alternative characterizations of the space $\mathcal{D}$ have been explored to tackle
this problem, for example the interpretation of elements in $\mathcal{D}$ as so-called
compositional data, or as elements of the Wasserstein-2 space. The latter
interpretation will be explored in Section~\ref{sec:fréchet regression} as this general
approach to regression can easily incorporate the geometry of a given space that is
implied by its metric (in this case, the Wasserstein-2 metric $d_\mathcal{W}$.)

Another approach is the general characterization of $\mathcal{D}$ as a nonlinear space
with certain constraints that can be done away with through a preliminary transformation
to a linear representation space. The compositional data approach can be seen as a
special case of this general transformation approach described in \textcite{PetersenMüller2016}.
The special case of this approach that we will explore in the following section is the
log quantile density (LQD) transformation.

\subsubsection{LQD Transformation}
\label{sec:transformation_interpretation}
Instead of the approach above, one can recognize the nonlinear character of the density
space and accomodate the use of fda methods such that it is respected. Several authors
have proposed a prior transformation of the densities into a linear space
\parencites[e.g.][]{Hron2016}[][]{PetersenMüller2016} where one can then work with the
usual fda methods on the transformed densities, and then later transform the results
back to density space for interpretation and visualization purposes.

Following the argument in \textcite{KokoszkaEtAl2019}, we use the method first outlined
in \textcite{PetersenMüller2016} and use the log quantile density (LQD) transformation
$\psi : \nobreak \mathcal{D} \to \mathcal{L}^2([0,1])$, defined by
\begin{equation}
    \label{eq:lqd_definition}
    \psi (f)(u) = \log q(u) = -\log f(Q(u)), \quad u \in [0,1],
\end{equation}
to transform the pdfs from the constrained space $\mathcal{D}$ into the "unconstrained"
space $\mathcal{L}^2([0,1])$.\footnote{We extend our usage of the argument $u$ to this
function because this transformed density is by definition on the same support as are
qfs and qdfs.} The intuition is that, from the functions given in
Definition~\ref{def:distributionfuncs}, the qf and qdf have the least constraints ---
only requiring that $q = \nobreak Q' \geq \nobreak 0$. Thus by working with the logarithm
of the qdf, one deals with a function that has no more constraints \parencite[cf.][]{KokoszkaEtAl2019}.

However, since the data generating process (DGP) that we will use in our simulation study
produces densities that are on different supports, we violate the assumption of common
compact support of the densities in the density sample that \textcite{PetersenMüller2016}
need.\footnote{We also violate this assumption because of our method of dealing with
numerical issues, see Appendix~\ref{sec:numerics}.} This leads to problems since the information
about location of support is lost when transforming a pdf to a qdf.\footnote{Sligthly
paraphrasing \textcite{KokoszkaEtAl2019}, the transformation from $f$ with quantile
density $q$ to its LQD is not invertible, since for any constant $c$, the quantile density of
$f(\cdot - c)$, the density $f$ shifted by $c$, is also $q$.}
Hence we use the slightly modified transformation and inverse transformation described
by \textcite{KokoszkaEtAl2019}. In this paper a very general transformation method is
described for the case of different supports. Denote by $\mathcal{A}_i$ the support of
the density $f_i$. For generality, we assume
\begin{equation*}
    \exists i, j \, : \, \mathcal{A}_i \cap \mathcal{A}_j = \emptyset,
\end{equation*}
i.e., we do not want to assume a common "anchor" point in the supports of the $f_i$. So
we use the following most general transformation proposed by the authors:
\begin{equation}
    T : \mathcal{D} \to \mathcal{L}^2([0,1]) \times \mathcal{A}, \text{ given by } T(f) = (\psi(f), Q(u^*)),
\end{equation}
where $\mathcal{A}$ is the support of density $f$, $Q$ is the corresponding quantile function,
and $u^* \in [0,1]$ is a fixed percentile level for which we save the value of the support
of $f$. Since we deal with a compact support, we can\footnote{\textcite{KokoszkaEtAl2019}
excluded 0 and 1 as possible values for $u^*$. This could have been motivated by having
a possibly unbounded support of the pdfs which could yield $Q(0)$ or $Q(1)$ to be
infinity, so useless for the transformation method. But this must have been a typo,
since they also only considered densities with bounded support, as defined in their
appendix.} choose the boundary value 0, which is most intuitive: we save the left bound
of the support of $f$, and for the inverse transformation we add that bound when computing
the quantile function $Q$ from the LQD. The (forward) transformation method is described
in Algorithm~\ref{alg:forward}, and the inverse transformation in Algorithm~\ref{alg:backward}.

\begin{algorithm}
    \caption{Forward transformation}
    \label{alg:forward}
    \begin{algorithmic}[1]
    \Require Data pairs \( (x_l, f(x_l)), l = 0, \ldots, L \)
    \Ensure Data pairs \( (u_l, \psi(f)(u_l)), l = 0, \ldots, L \) and \( c \)
    \For{\( l = 1, \ldots, L \)}
        \State Compute \( F(x_l) = \int_{x_0}^{x_l} f(x) \, dx \) by numerical integration of the pairs \( (x_l, f(x_l)) \)
        \State Define \( u_l = F(x_l) \) so that \( Q(u_l) = x_l \)
        \State Compute \( \psi(f)(u_l) = -\ln(f(x_l)) \)
        \State Find \( j \) such that \( x_j = 0 \) and set \( c = Q(0) \)
    \EndFor
    \end{algorithmic}
\end{algorithm}

\begin{algorithm}
    \caption{Backward transformation}
    \label{alg:backward}
    \begin{algorithmic}[1]
    \Require Data pairs \( (u_l, g(u_l)), l = 0, \ldots, L \) and \( c \)
    \Ensure Data pairs \( (x_l, f(x_l)), l = 0, \ldots, L \)
    \For{\( l = 1, \ldots, L \)}
        \State Compute \( Q(u_l) = c + \int_{u^*}^{u_l} \exp\{g(s)\} \, ds \) by numerical integration of the pairs \( (u_l, \exp\{g(u_l)\}) \)
        \State Define \( x_l = Q(u_l) \) so that \( F(x_l) = u_l \)
        \State Compute \( f(x_l) = \exp \{-g(u_l)\} \)
    \EndFor
    \end{algorithmic}
\end{algorithm}

Here, $g$ is some function from the transform space $\mathcal{L}^2([0,1])$,
resulting from some calculations we performed with the transformed densities. For
the computation of the inverse, we need to supply a value of $c$ for this function $g$.
In our application we will deal with two kinds of functions $g$: estimated LQDs from
our functional regression for some grid of predictor values; and a mean LQD computed from
the LQD sample. For the former, we estimated $c$ by defining a function $c(p)$, which
assigns each $c$ to the predictor value $p$ with which it was observed, and calculates $\hat{c}$
via linear interpolation. For the mean LQD, we computed the mean of the $c$ values in the sample
and used that as an estimator. These $\hat{c}$ are supposed to estimate the left bound
of the support of the resulting density when performing the inverse transform $T^{-1}$
described in \ref{alg:backward}.

\begin{figure}[h!]
    \centering
    \begin{subfigure}[b]{0.9\textwidth}
        \centering
        \resizebox{\linewidth}{!}{\input{../bld/figures/fda/naive_trunc_rep_vs_orig.pgf}}
        \caption[Truncated representation --- naive]{Naive method.}
        \label{fig:naive_trunc_rep}
    \end{subfigure}
    \hfill % add some horizontal spacing
    \begin{subfigure}[b]{0.9\textwidth}
        \centering
        \resizebox{\linewidth}{!}{\input{../bld/figures/fda/trunc_rep_vs_orig.pgf}}
        \caption[Truncated representation --- LQD method]{LQD method.}
        \label{fig:trunc_rep}
    \end{subfigure}
    \caption[Truncated Karhunen-Loève representations]{Truncated Karhunen-Loève
    representations of a sample density. The blue and orange curves result from using one
    and two principal components, respectively. The black curve is the sample density
    that is to be represented. We added a small perturbation to the orange curve in the
    naive method to better differentiate it from the blue curve.}
    \label{fig:trunc_reps}
\end{figure}

As an illustration for the difference in quality of the LQD method to the naive method,
Figure~\ref{fig:trunc_reps} shows the truncated Karhunen-Loève representations of a
sample density with both methods. One can clearly see that the representations based on
the LQD method produce much better fitting curves than the naive method. Further
illustrations, as well as the description of the DGP behind this density, are given in
Appendix~\ref{sec:illustration_lqd}.

In the following section, we will explain a functional regression model based on the
LQD method.

\subsection{Functional Regression Model}
\label{sec:func_reg_model}
The transformation approach employed by \textcite{TalskaEtAl2018} can be considered a
special case of the general transformation approach explained by \textcite{PetersenMüller2016}
\parencite[cf.][]{PetersenZhangKokoszka2022}. So we follow their modelling of a functional
response model for our LQD transformation method.

\subsubsection{Construction of Model}
\label{sec:model_construction}
In the following, $n$ denotes the sample size and $r$ denotes the number of scalar
predictors $p$. Our original regression model of interest is
\begin{equation}
    \label{eq:dens_reg_model}
    f_i(x) = \beta_0(x) + \sum_{j=1}^{r} p_{ij} \beta_j(x) + \varepsilon_i(x), \quad i = 1, \ldots, n,
\end{equation}
where $f_i$ is the $i$-th observed density, $p_{ij}$ the $i$-th observed value of the
$j$-th predictor, $\varepsilon_i$ is an i.i.d. functional random error in $\mathcal{L}^2([0,1])$
with mean zero, and lastly $\beta_j$ is the $j$-th functional regression coefficient of the $j$-th
predictor $p_j$. This can be written simpler in matrix notation as
\begin{equation}
    \label{eq:dens_reg_model_matrix}
    \mathbf{f}(x) = \mathbf{P} \boldsymbol{\beta}(x) + \boldsymbol{\varepsilon}(x).
\end{equation}
where $\mathbf{f}$ is the $n$-dimensional vector of observed pdfs, $\mathbf{P}$ is an $n \times (r+1)$
matrix containing the scalar predictor values $p_{ij}$ (and the first column containing only
ones), $\boldsymbol{\beta}$ is the $(r+1)$-dimensional vector of functional regression coefficients,
and $\boldsymbol{\varepsilon}$ is the $n$-dimensional vector of functional random errors.

Applying the LQD transformation \eqref{eq:lqd_definition} to \eqref{eq:dens_reg_model}
yields
\begin{equation}
    \label{eq:lqd_reg_model}
    \psi(f_i)(u) = \psi(\beta_0)(u) + \sum_{j=1}^{r} p_{ij} \psi(\beta_j)(u) + \psi(\varepsilon_i)(u), \quad i = 1, \ldots, n.
\end{equation}
Defining $\psi(\boldsymbol{\beta})$ as the $(r+1)$-dimensional vector of transformed
regression coefficients, we estimate $\psi(\boldsymbol{\beta})$ by minimizing the sum of
squared errors:
\begin{equation}
    \psi(\hat{\boldsymbol{\beta}})(u)
    \coloneqq \argmin_{\beta_0,\dots, \beta_r} \sum_{i=1}^{n} \|\psi(\varepsilon_i) (u)\|_2^2
    = \int \sum_{i=1}^{n} \psi(\varepsilon_i)^2 (u) \, du
    = \sum_{i=1}^{n} \left( \psi(f_i)(u) - \sum_{j=0}^{r} p_{ij} \beta_j(u) \right)^2,
\end{equation}
where $p_{i0} = 1 \ \forall i$, and the integral and summation operations have been
exchanged in the first equality because they are both linear operators
\parencites[cf.][Chapter~13.4]{RamsaySilverman2005}[][Chapter~5.1]{KokoszkaReimherr2017}.

\subsubsection{Computational Details}
\label{sec:com_details_func_reg}
Since there are no particular restrictions on the functions in the transformed space,
straightforward pointwise minimization can be performed by calculating
\begin{equation}
    \psi(\hat{\boldsymbol{\beta}})
    = \left(\mathbf{P}^T \mathbf{P}\right)^{-1} \mathbf{P}^T \psi(\mathbf{f}).
\end{equation}
Practically, we observed $\psi(\mathbf{f})$ on a dense grid of evaluation points and
accordingly computed a discretized version of $\psi(\hat{\boldsymbol{\beta}})$ with the
above method. Using linear interpolation between these values, we obtained an approximation
of $\psi(\hat{\boldsymbol{\beta}})$.

We obtain the estimated pdf for any particular predictor values $p_1, \dots, p_r$ by
calculating
\begin{equation}
    \psi(\hat{\boldsymbol{\beta}})
    = \left(\mathbf{P}^T \mathbf{P}\right)^{-1} \mathbf{P}^T \psi(\mathbf{f}).
\end{equation}

We will present this regression method in section \ref{sec:application} where we will
compare it against the Fréchet regression method. To that end, we will explain the
Fréchet regression model from \textcite{PetersenMüller2019} in the following section.
