This section will introduce basic fda concepts and methods used throughout this thesis.

\subsection{Covariance Operator and Covariance Fuctions}
\label{sec:covs}

\subsection{Functional Principal Component Analysis}
\label{sec:fpca}
The most popular method for describing structure in our functional data is Functional
Principal Component analysis (FPCA). This is an analogue to Principal Component Analysis
(PCA) from multivariate statistics in the case of infinite dimensions. It builds on the
Karhunen-Loève decomposition (described in the section \ref{sec:mercer and kh}) to
recover functions that describe the main modes of variation, in descending order.

We computed the discretized grid as described in \citet[Chapter~8.4.1]{RamsaySilverman2005}. Another analogous way
is described in \textcite{KneipUtikal2001}. (\textcite{Delicado2011} SUMMARIZES THIS.)

\subsection{Descriptive Methods}
\label{sec:fpca_descriptives}

\subsubsection{Modes of Variation}
\label{sec:modes_of_variation}