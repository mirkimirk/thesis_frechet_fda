We see a more continous estimate in the Fréchet regression. Clearly the mean shifts
steadily to the right, as x increases, as well as more variance.

Because in the functional regression on the log transformed qds, we have to adjust with
the range with the individual quantile of the observation, and otherwise there is no
possibility to keep the mean, the linear dependence of the mean is lost. One can though
see that the variance rises with x. This information is not lost in the log qdfs.

\begin{table}[h]
    \centering
    \begin{tabular}{lccc}
        \hline
        ISE & $n = 50$ & $n = 100$ & $n = 200$ \\
        \hline
        Fréchet             & 0.18 & 0.08 & 0.04 \\
        Fréchet, D.-Est.    & 32.31 & 27.98 & 27.38 \\
        LQD                 & 34.49 & 31.19 & 31.88 \\
        LQD, D.-Est.        & 47.08 & 43.39 & 43.43 \\
        \hline
    \end{tabular}
    \caption{Mean ISE Values for different methods and scenarios. "D.-Est." indicates
    a scenario with previous density estimation procedure.}
\end{table}

Moreover, the quality of the density estimates deteriorates with sample size, suggesting
to vary the bandwidth choice with sample size, as \textcite{PetersenMüller2019} did in
their comparison of the Fréchet estimator with the Nadaraya-Watson estimator. Due to
time constraints and the computational requirements, we did not conduct such
hyperparameter tuning but chose a rule-of-thumb bandwidth
\parencites[Chapter~3.4.1]{Silverman1986}[Chapter~2.2.1]{LiRacine2007}

The curve for the Fréchet mean with observed densities is farther off than the Fréchet
mean with a previous density estimation step. We think this must be because the densities
yielded by the density estimation have a smaller variance in the left bounds of their
support, so that the uncertainty introduced by the estimation of the new left bound for
the average is smaller.

\subsection{Results}
\label{sec:sim_results}
In this section, we briefly visualize and discuss the results from the simulation study.

\begin{figure}[h]
    \centering
    % \input{../bld/figures/frechet/ise_frechet.pgf}
    \resizebox{1\textwidth}{!}{\input{../bld/figures/frechet/ise_frechet.pgf}}
    \caption[Simulation results: ISE boxplot --- Fréchet method with observed densities]{Boxplot showing distribution of
    ISE from Fréchet estimates in scenario with directly observed densities.}
    \label{fig:ise_frechet}
\end{figure}

Figure \ref{fig:ise_frechet} shows the distribution of ISE values for the Fréchet method,
in the scenario where the densities are directly observed. One can clearly see that, with
increasing sample size, they become lower and have a lower variance, with less outliers
outside the whiskers. This is as expected, since increasing the sample size
means more information to use for estimation, so the predictions should be closer to
their mean (the true value), and the variance of the predictions around their mean should
get smaller.

\begin{figure}[h]
    \centering
    % \input{../bld/figures/frechet/ise_frechet_denstimation.pgf}
    \resizebox{1\textwidth}{!}{\input{../bld/figures/frechet/ise_frechet_denstimation.pgf}}
    \caption[Simulation results: ISE boxplot --- Fréchet method with estimated densities]{Boxplot
    showing distribution of ISE from Fréchet estimates in scenario with additional
    density estimation step to obtain densities.}
    \label{fig:ise_frechet_denstimation}
\end{figure}

Figure \ref{fig:ise_frechet_denstimation} shows the distribution of ISE values for
the Fréchet method, in the scenario with a previously included density estimation step.
Here, the ISE values are much higher, and increasing sample sizes yields only minor
improvement. This hints at a problem in the density estimation method. Nonetheless,
the values are better than the ISE values for the transformation method in the same
scenario.

\begin{figure}[h]
    \centering
    % \input{../bld/figures/frechet/ise_func_reg_both.pgf}
    \resizebox{\textwidth}{!}{\input{../bld/figures/frechet/ise_func_reg_both.pgf}}
    \caption[Simulation results: ISE boxplot --- functional regression]{Boxplot showing
    distribution of ISE from functional regression with LQD method in both scenarios.}
    \label{fig:ise_func_reg_both}
\end{figure}

Finally, Figure \ref{fig:ise_func_reg_both} shows the distribution of ISE values for
the functional regression method, in both scenarios and for different sample sizes. One 
can see that the added uncertainty by the density estimation step lowers the convergence
rate, and that in both scenarios the ISE gets stuck on a high level as increasing the
sample size from 100 to 200 makes negligible difference. This is probably
because of additional tuning that has to be done in the density estimation step (since
this scenario makes the Fréchet method also signifiantly worse), such as a better method
of bandwidth choice. The big difference between the ISE boxplots for the Fréchet method and
the transformation method even in the case of directly observed densities must be explained 
by the beforementioned (see Section~\ref{sec:transformation_interpretation}) need to
estimate the location of the support in the transformation method since this information
is lost when transforming the pdf to the qdf. Maybe one can improve upon the linear
interpolation method to compute the left bounds.