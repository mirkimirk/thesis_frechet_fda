In this section, we show how the LQD method and Fréchet method work in practice and
compare their performance in a simulation study. First, we will describe the DGP that
will produce the data for our illustrations and 

\subsection{Data Generating Process}
\label{sec:app_dgp}
The DGP used here is the same as \textcite{PetersenMüller2019} used in their simulation


\subsection{Illustration of Methods in Example}
\label{sec:app_illustration}

\subsubsection{LQD method}
\label{sec:app_lqd}

\begin{figure}[h!]
    \centering
    % \input{../bld/figures/frechet/betas.pgf}
    \resizebox{1\textwidth}{!}{\input{../bld/figures/frechet/betas.pgf}}
    \caption[Estimated betas in LQD functional regression]{Plot of estimated betas from
    functional regression with LQD method.}
    \label{fig:betas}
\end{figure}
Figure~\ref{fig:betas} shows the estimated $\psi(\hat{\boldsymbol{\beta}})$ values. $\psi(\hat{\beta_1})$
is estimated to be a constant greater than one, meaning that $\hat{\beta}_1$ is a positive
constant after back transformation, reflecting the positive linear effect that $p$ has
on the responses. $\psi(\hat{\beta}_0)$ reflects the mean LQD, or $\hat{\beta}_0$ reflects
the mean pdf. Calculating fitted values via $\psi(\hat{f}) = \psi(\hat{\beta}_0) + \psi(\hat{\beta}_1) p$
means that we are adding an increasing constant for increasing $p$ to a (log) qdf, and
this constant makes the qdf values uniformly larger everywhere, thus making them relatively
more even. A flatter qdf means more variance so we should see a gradual increase in
variance  in the plots of the estimated pdfs via LQD method. The location, as mentioned
in Section~\ref{sec:transformation_interpretation}, has to be estimated separately and is not
contained in the information of qdfs, so also not in the estimated LQDs or regression
coefficients.
\begin{figure}[h!]
    \centering
    % \input{../bld/figures/frechet/beta0vsmean.pgf}
    \resizebox{1\textwidth}{!}{\input{../bld/figures/frechet/beta0vsmean.pgf}}
    \caption[Equality of mean LQD function and $\psi(\hat{\beta}_0)$]{Plot that shows the equality
    of the mean LQD function and $\beta_0$. Some perturbation was added so one can
    distinguish the curves.}
    \label{fig:beta0vsmean}
\end{figure}
To show that $\psi(\hat{\beta}_0)$ and the (cross-sectional) mean LQD function are the
same, we printed Figure~\ref{fig:beta0vsmean}. Next, we will turn to the estimates of
the densities obtained via LQD method and compare them with the densities produces by
the true stochastic process.
\begin{figure}[h!]
    \centering
    % \input{../bld/figures/frechet/func_est_vs_true.pgf}
    \resizebox{1\textwidth}{!}{\input{../bld/figures/frechet/func_est_vs_true.pgf}}
    \caption[Comparison: estimated vs. true densities --- LQD]{Estimated densities from
    functional regression with LQD method plotted against the true densities for some
    selected predictor values.}
    \label{fig:func_est_vs_true}
\end{figure}
Figure~\ref{fig:func_est_vs_true} plots the estimated densities against their true
counterparts. We see that the general trend to higher variance is well reflected in the
estimated curves

Because in the functional regression on the log transformed qds, we have to adjust with
the range with the individual quantile of the observation, and otherwise there is no
possibility to keep the mean, the linear dependence of the mean is lost. One can though
see that the variance rises with x. This information is not lost in the log qdfs.

\subsubsection{Fréchet method}
\label{sec:app_frechet}

\begin{figure}[h!]
    \centering
    % \input{../bld/figures/frechet/1stlastqf.pgf}
    \resizebox{1\textwidth}{!}{\input{../bld/figures/frechet/1stlastqf.pgf}}
    \caption[Two sample qfs]{Plot of two sample qfs, one with predictor value $p = -1$,
    and the other for $p = 1$.}
    \label{fig:1stlastqf}
\end{figure}

Figure \ref{fig:1stlastqf} shows two example qfs, generated from the above mentioned
process with the predictor values $p=-1$ and $p=1$. One can see that both location and
scale are much larger for the curve with the higher predictor value.

\begin{figure}[h!]
    \centering
    % \input{../bld/figures/frechet/some_densities.pgf}
    \resizebox{1\textwidth}{!}{\input{../bld/figures/frechet/some_densities.pgf}}
    \caption[Several sampled pdfs]{Plot of several sample pdfs.}
    \label{fig:some_densities}
\end{figure}

Figure \ref{fig:some_densities} plots the corresponding pdfs of some of the generated qfs.

\begin{figure}[h!]
    \centering
    % \input{../bld/figures/frechet/frechet_estimates.pgf}
    \resizebox{1\textwidth}{!}{\input{../bld/figures/frechet/frechet_estimates.pgf}}
    \caption[Estimated pdfs from Fréchet method]{Plot of estimated densities via Fréchet regression method.
    The densities from left to right are corresponding to increasing predictor values $p$ on an
    equidistant grid on $[-1,1]$ with 50 grid points.}
    \label{fig:frechet_estimates}
\end{figure}

\begin{figure}[h!]
    \centering
    % \input{../bld/figures/frechet/frechet_estimates_3d.pgf}
    \resizebox{1\textwidth}{!}{\input{../bld/figures/frechet/frechet_estimates_3d.pgf}}
    \caption[Estimated qfs from Fréchet method]{Plot of estimated qfs via Fréchet regression method.}
    \label{fig:frechet_estimates_3d}
\end{figure}

Figures \ref{fig:frechet_estimates} and \ref{fig:frechet_estimates_3d} plot the Fréchet
estimates of the pdfs and qfs respectively. In the first figure, the estimated pdfs for
different predictor values are drawn on top of each other, and in the second, a 3d plot
was used to visualize the change of shape of the qfs with a One can see a smooth
increase in location and variance in both figures with an increase in the predictor
value.

% \begin{figure}[h!]
%     \centering
%     % \input{../bld/figures/frechet/all_observations_3d.pgf}
%     \resizebox{1\textwidth}{!}{\input{../bld/figures/frechet/all_observations_3d.pgf}}
%     \caption{Plot of sampled qfs and their predictor values.}
%     \label{fig:all_observations_3d}
% \end{figure}

\begin{figure}[h!]
    \centering
    % \input{../bld/figures/frechet/frechet_est_vs_true.pgf}
    \resizebox{1\textwidth}{!}{\input{../bld/figures/frechet/frechet_est_vs_true.pgf}}
    \caption[Comparison: estimated vs. true densities --- Fréchet]{Estimated
    densities from Fréchet regression plotted against the true densities for some selected
    predictor values.}
    \label{fig:frechet_est_vs_true}
\end{figure}
Figure~\ref{fig:frechet_est_vs_true} shows a more continuous estimate in the Fréchet
regression. Clearly the mean shifts steadily to the right, as x increases, as well as
more variance.

\subsection{Results}
\label{sec:app_results}
In this section, we briefly visualize and discuss the results from the simulation study.

\begin{figure}[h!]
    \centering
    % \input{../bld/figures/frechet/ise_frechet.pgf}
    \resizebox{1\textwidth}{!}{\input{../bld/figures/frechet/ise_frechet.pgf}}
    \caption[Simulation results: ISE boxplot --- Fréchet method with observed densities]{Boxplot showing distribution of
    ISE from Fréchet estimates in scenario with directly observed densities.}
    \label{fig:ise_frechet}
\end{figure}

Figure \ref{fig:ise_frechet} shows the distribution of ISE values for the Fréchet method,
in the scenario where the densities are directly observed. One can clearly see that, with
increasing sample size, they become lower and have a lower variance, with less outliers
outside the whiskers. This is as expected, since increasing the sample size
means more information to use for estimation, so the predictions should be closer to
their mean (the true value), and the variance of the predictions around their mean should
get smaller.

\begin{figure}[h!]
    \centering
    % \input{../bld/figures/frechet/ise_frechet_denstimation.pgf}
    \resizebox{1\textwidth}{!}{\input{../bld/figures/frechet/ise_frechet_denstimation.pgf}}
    \caption[Simulation results: ISE boxplot --- Fréchet method with estimated densities]{Boxplot
    showing distribution of ISE from Fréchet estimates in scenario with additional
    density estimation step to obtain densities.}
    \label{fig:ise_frechet_denstimation}
\end{figure}

Figure \ref{fig:ise_frechet_denstimation} shows the distribution of ISE values for
the Fréchet method, in the scenario with a previously included density estimation step.
Here, the ISE values are much higher, and increasing sample sizes yields only minor
improvement. This hints at a problem in the density estimation method. Nonetheless,
the values are better than the ISE values for the transformation method in the same
scenario.

\begin{figure}[h!]
    \centering
    % \input{../bld/figures/frechet/ise_func_reg_both.pgf}
    \resizebox{\textwidth}{!}{\input{../bld/figures/frechet/ise_func_reg_both.pgf}}
    \caption[Simulation results: ISE boxplot --- functional regression]{Boxplot showing
    distribution of ISE from functional regression with LQD method in both scenarios.}
    \label{fig:ise_func_reg_both}
\end{figure}

Finally, Figure \ref{fig:ise_func_reg_both} shows the distribution of ISE values for
the functional regression method, in both scenarios and for different sample sizes. One
can see that the added uncertainty by the density estimation step lowers the convergence
rate, and that in both scenarios the ISE gets stuck on a high level as increasing the
sample size from 100 to 200 makes negligible difference. This is probably
because of additional tuning that has to be done in the density estimation step (since
this scenario makes the Fréchet method also signifiantly worse), such as a better method
of bandwidth choice. The big difference between the ISE boxplots for the Fréchet method and
the transformation method even in the case of directly observed densities must be explained
by the beforementioned (see Section~\ref{sec:transformation_interpretation}) need to
estimate the location of the support in the transformation method since this information
is lost when transforming the pdf to the qdf. An area of improvement could be to use
more sophisticated estimation methods than linear interpolation to compute the left bounds
of the supports.

\begin{table}[h!]
    \centering
    \begin{tabular}{lccc}
        \hline
        ISE & $n = 50$ & $n = 100$ & $n = 200$ \\
        \hline
        Fréchet             & 0.18 & 0.08 & 0.04 \\
        Fréchet, D.-Est.    & 32.31 & 27.98 & 27.38 \\
        LQD                 & 34.49 & 31.19 & 31.88 \\
        LQD, D.-Est.        & 47.08 & 43.39 & 43.43 \\
        \hline
    \end{tabular}
    \caption{Mean ISE Values for different methods and scenarios. "D.-Est." indicates
    a scenario with previous density estimation procedure.}
    \label{tab:mean_ise_values}
\end{table}
Table~\ref{tab:mean_ise_values} shows the mean ISE values from the different simulation
settings, for both methods. They reflect the same findings discussed in the boxplots
above.
Moreover, the quality of the density estimates deteriorates with sample size, suggesting
to vary the bandwidth choice with sample size, as \textcite{PetersenMüller2019} did in
their comparison of the Fréchet estimator with the Nadaraya-Watson estimator. Due to
time constraints and the computational requirements, we did not conduct such
hyperparameter tuning but chose a rule-of-thumb bandwidth
\parencites[Chapter~3.4.1]{Silverman1986}[Chapter~2.2.1]{LiRacine2007}

The curve for the Fréchet mean with observed densities is farther off than the Fréchet
mean with a previous density estimation step. We think this must be because the densities
yielded by the density estimation have a smaller variance in the left bounds of their
support, so that the uncertainty introduced by the estimation of the new left bound for
the average is smaller.